\chapter{Project 3: Virtual Memory}

\section{Overview}

By now your OS can load multiple user applications at once, it can service their requests for
accessing system resources and for managing processes and threads. However, the number and size of
programs that can run is limited by the machine's main memory size. In this assignment, you will
remove that limitation.

You will build this assignment on top of the last one. Test programs from project 2 should also work
with project 3. You should take care to fix any bugs in your project 2 submission before you start
work on project 3, because those bugs will most likely cause the same problems in project 3.

\projectname already supports some virtual memory features such as lazy mapping and memory mapped
files and makes extensive use of them already, so you won't have to intervene in those areas.
However, you will have to implement the system calls to allow a user-application to dynamically
allocate and free virtual memory - this could allow a user application to implement a heap allocator.
Moreover, this memory must be shareable by any number of processes.

You will also have to implement per process quotas: you should limit the number of physical frames
a process uses and the number of open files a process can have at once.

And finally you will implement a swapping mechanism which will cause frame eviction to occur either 
when there are no more free frames in physical memory or when a process reaches its frame quota.

\subsection{Memory Terminology}
Careful definitions are needed to keep discussion of virtual memory from being confusing. Thus, we
begin by presenting some terminology for memory and storage.

\subsubsection{Pages}

A page, sometimes called a virtual page, is a continuous region of virtual memory 4,096 bytes (the 
page size) in length. A page must be page-aligned, that is, start on a virtual address evenly
divisible by the page size. A 64-bit virtual addresses can be divided into 6 sections as illustrated
below:
\begin{itemize}
	\item The most significant 16 bits are unused because they reflect the 47th bit.
	\item The next 4 sections of 9 bits each provide an index into to the corresponding paging table
structure.
	\item The final 12 bits provide the offset within the final physical address.
\end{itemize}

\begin{verbatim}
63             48 47     39 38     30 29     21 20     12 11          0
+----------------+---------+---------+---------+---------+------------+
| Unused         |  PML4   | Dir Ptr |   Dir   |  Table  |   Offset   |
+----------------+---------+---------+---------+---------+------------+
                             Virtual Address
\end{verbatim}

Each process has an independent set of user (virtual) pages, which are those pages below virtual 
address 0x8'0000'0000'0000 (128 TiB), while the kernel virtual space begins at 
\var{gVirtualToPhysicalOffset} which is typically 0xFFFF'8000'0000'0000 (almost 16 EiB). The set of
kernel (virtual) pages, on the other hand, is global, remaining the same regardless of what thread  
or process is active. The kernel may access both user and kernel pages, but a user process may 
access only its own user pages. See \fullref{sect:VirtMemLayout} for more information.

\projectname provides several useful functions for working with virtual addresses. See 
\fullref{sect:VirtAddr} for details.

\subsubsection{Frames}

A frame, sometimes called a physical frame or a page frame, is a continuous region of physical
memory. Like pages, frames must be page-size and page-aligned. Even if the processor runs in 64-bits
mode the maximum physical address is not $2^{64}$ (16 EiB), the maximum addressable physical address
differs from CPU to CPU (this value can be found out by querying a CPUID leaf), however, according 
to the Intel manual the maximum physical address size is limited to $2^{52}$ bits (4 PiB).

Thus, a 52-bit physical address can be divided into a 40-bit frame number and a 12-bit frame offset,
(or just offset) like this:

\begin{verbatim}
63    52                  12 11         0
+-------+-------------------+-----------+
| 00000 |      Frame Number |   Offset  |
+-------+-------------------+-----------+
              Physical Address
\end{verbatim}

When paging is enabled the x86 architecture works with virtual addresses, transparently accessing
the physical memory mapped by the address. Thus, the software executing does not need to know the
actual whereabouts of the memory or the memory topology found in the system,

\projectname provides functions for translating between physical addresses and kernel virtual
addresses. See \fullref{sect:VirtAddr} for details.

\subsubsection{Page Tables}

The x86 processors translate virtual addresses to physical addresses through the use of some hardware
defined structures, called paging tables. These are hierarchical structures which describe the
virtual address space and provide the final physical address, they also specify the access rights
(read/write/execute) and the privilege level required (kernel or user-mode access). \projectname
provides page table management code in \file{pte.h}. See \fullref{sect:PageTables} for more
information.

The diagram below illustrates the relationship between pages and frames. The virtual address, on the
left, consists of 4 page indexes (one for each paging level) and an offset. The paging tables
translate the page indexes into a frame number, which is combined with the unmodified offset to 
obtain the physical address, on the right.

\begin{verbatim}
                      +-------------+
    .---------------->|Paging Tables|---------.
    |                 +-------------+         |
47  |    12 11      0                  52     V   12 11      0
+----------+--------+                  +------------+--------+
| Page Idx | Offset |                  | Frame No   | Offset |
+----------+--------+                  +------------+--------+
 Virt Addr     |                              Phys Addr ^
               \_______________________________________/
\end{verbatim}

A more detailed illustration is given in \fullref{fig:PageTranslation}, here, what was previously
called "Page Idx" is now properly separated into its 4 parts: the index in the PML4 table, the index
in the PDPT table, the index in the PD table and the index in the PT table.

\begin{figure}
\begin{verbatim}
47      39 38     30 29     21 20     12 11          0
+---------+---------+---------+---------+------------+
| PML4 Idx| PDPT Idx| PD Idx  | PT Idx  | Page Offset|
+---------+---------+---------+---------+------------+
    |          |        |         |            |___________
____/          |        \_____    \__________              \
/              |              \              \              \
/      PML4    |      PDPT    |       PD     |       PT     |       Data Page
/    ._______. |    ._______. |    ._______. |    ._______. |    .____________.
| 511|_______| | 511|_______| | 511|_______| | 511|_______| |    |____________|
| 510|_______| | 510|_______| | 510|_______| | 510|_______| |    |____________|
| 509|_______| | 509|_______| | 509|_______| | 509|_______| |    |____________|
| 508|_______| | 508|_______| | 508|_______| | 508|_______| |    |____________|
|    |       | |    |       | |    |       | |    |       | |    |            |
|    |       | \___\|   .   | \___\|   .   | \___\|   .   | \___\|     .      |
|    |   .   |     /|   .   |     /|   .   |     /|   .   |     /|     .      |
\___\|   .   |_     |   .   |_     |   .   |_     |   .   |_     |     .      |
    /|   .   | \    |   .   | \    |   .   | \    |   .   | \    |     .      |
     |   .   | |    |   .   | |    |   .   | |    |   .   | |    |     .      |
     |       | |    |       | |    |       | |    |       | |    |            |
     |_______| |    |_______| |    |_______| |    |_______| |    |____________|
    4|_______| |   4|_______| |   4|_______| |   4|_______| |    |____________|
    3|_______| |   3|_______| |   3|_______| |   3|_______| |    |____________|
    2|_______| |   2|_______| |   2|_______| |   2|_______| |    |____________|
    1|_______| |   1|_______| |   1|_______| |   1|_______| |    |____________|
    0|_______| \__\0|_______| \__\0|_______| \__\0|_______| \__\ |____________|
                  /              /              /              /
\end{verbatim}
		\caption{Detailed paging}
	\label{fig:PageTranslation}
\end{figure}

\subsubsection{Swap Slots}

A swap slot is a continuous, page-size region of disk space in the swap partition. Although hardware
limitations dictating the placement of slots are looser than for pages and frames, swap slots should
be page-aligned because there is no downside in doing so.

\subsection{Memory Management Initialization}

Before reading this section, you should first read \fullref{sect:MemManagement}.

The initialization of the memory subsystem begins in \func{MmuInitSystem} which aggregates two
other subsystems: physical memory subsystem (PMM) and virtual memory subsystem (VMM). Once these
systems are initialized two heap allocators are built on top of them.

The PMM will be the first one initialized by a call to \func{PmmInitSystem}, this function uses the
information provided by the firmware to determine how much physical memory the system possesses and
which frames are free for use. A bitmap is then initialized to track the usage of physical frames,
however no additional information is kept except if the frame is available or not. Because the size
of physical memory doesn't change after boot and the size of a frame is fixed to 4 KiB in size a
bitmap is perfect for this job.

Once this function returns, physical frames can be allocated using \func{PmmReserveMemoryEx} and
freed using \func{PmmReleaseMemory}: these functions simply flip some bits in the bitmap to mark if
a frame is free or taken.

Now, some continuous physical frames are allocated for the new paging table structures, this is done
in \func{\_MmuInitPagingSystem}, these structures will be mapped in virtual space at
\var{gVirtualToPhysicalOffset} (0xFFFF'8000'0000'0000) + \textit{PHYSICAL\_ADDRESS}. Because these
structures are found at continous physical addresses and have a formula to determine their virtual
address the page manipulation algorithms are simplified.

The VMM is now initialized through a call to \func{VmmInit}, this initializes the virtual address
space for the system process, once this call finishes \func{VmmAllocRegionEx} and 
\func{VmmFreeRegionEx} can be used. The VMM manages the address space for each process in the sense
that it holds information about all the virtual memory it committed or reserved - this is required
for implementing lazy mapping and memory mapped files.

The kernel image will now be remapped using the new paging tables which were previously created, two
mappings are done: an identity mapping (this is required by the application processors when they
start-up and activate paging) and a mapping similar to the one used for the paging tables, where
the \macro{PA2VA} and \macro{VA2PA} macros can be used.

The change to the new paging tables is done through a simple MOV to CR3 assembly instruction using
the \func{\_\_writecr3} intrinsic function. After this change two heaps will be initialized, one
will be used for allocations created through \func{ExAllocatePoolWithTag}, the other is used
explicitly by \func{MmuReleaseMemory}.

After the application processors are woken up the kernel identity mappings are discarded by a call
to \func{MmuDiscardIdentityMappings}.

\subsection{Resource Management Overview}

You will need to design or augment the following data structures:

\begin{itemize}
	\item Supplemental page table
	
		Enables page fault handling by supplementing the hardware page table, see \fullref{sect:MngSpTable}.
		
	\item Frame table
	
		Allows efficient implementation of eviction policy, see \fullref{sect:MngFrameTable}.
		
	\item Swap table
	
		Tracks usage of swap slots,  see \fullref{sect:MngSwapTable}.
\end{itemize}

You do not necessarily need to implement three completely distinct data structures: it may be
convenient to wholly or partially merge related resources into a unified data structure.

For each data structure, you need to determine what information each element should contain. You
also need to decide on the data structure’s scope, either local (per-process) or global (applying to
the whole system), and how many instances are required within its scope.

Possible choices of data structures include arrays, lists, bitmaps, and hash tables. An array is
often the simplest approach, but a sparsely populated array wastes memory. Lists are also simple,
but traversing a long list to find a particular position wastes time. Both arrays and lists can be
re-sized, but lists more efficiently support insertion and deletion in the middle.

Although more complex data structures may yield performance or other benefits, they may also
needlessly complicate your implementation. Thus, we do not recommend implementing any advanced data
structure (e.g. a balanced binary tree) as part of your design.

\subsubsection{Managing the Supplemental Page Table}
\label{sect:MngSpTable}

The supplemental page table supplements the page table with additional data about each page. It is
needed because of the limitations imposed by the page table’s format. Such a data structure is often
called a "page table" also; we add the word "supplemental" to reduce confusion.

The supplemental page table is used for at least two purposes. Most importantly, on a page fault,
the kernel looks up the virtual page that faulted in the supplemental page table to find out what
data should be there. Second, the kernel consults the supplemental page table when a process
terminates, to decide what resources to free.

Because \projectname already supports lazy mappings and can handle page faults, such a table already
exits for each process, and the structure used to describe each mapping is \textit{VMM\_RESERVATION},
see \fullref{sect:VMM} for details.

You could use this already existing data structure and augment it with further information or you
could use a whole different structure.

\subsubsection{Managing the Frame Table}
\label{sect:MngFrameTable}

The frame table contains one entry for each frame that contains a user page. Each entry in the frame
table contains a pointer to the page, if any, that currently occupies it, and other data of your
choice. The frame table allows \projectname to efficiently implement an eviction policy, by choosing
a page to evict when no frames are free.

The frames are obtained by calling \func{PmmReserveMemory} or \func{PmmReserveMemoryEx} and are
freed using \func{PmmReleaseMemory}.

The most important operation on the frame table is obtaining an unused frame. This is easy when a
frame is free. When none is free, or when the process has reached its physical frame quota a frame
must be made free by evicting some page from its frame.

If no frame can be evicted without allocating a swap slot, but the swap is full, panic the kernel.
Real OSes apply a wide range of policies to recover from or prevent such situations, but these
policies are beyond the scope of this project.

The process of eviction comprises roughly the following steps:
\begin{enumerate}
	\item Choose a frame to evict, using your page replacement algorithm. The 'accessed' and
'dirty' bits in the page table, described in \fullref{sect:ADBits} will come in handy.

	\item Remove references to the frame from any page table that refers to it. Be careful, once you
have implemented shared memory multiple pages can refer to the same frame at a given time.

	\item If necessary, write the page to the file system or to swap.
\end{enumerate}
The evicted frame may then be used to store a different page.

\subsubsection{Managing the Swap Table}
\label{sect:MngSwapTable}

The swap table tracks in-use and free swap slots. It should allow picking an unused swap slot for
evicting a page from its frame to the swap partition. It should allow freeing a swap slot when its
page is read back or the process whose page was swapped is terminated.

You may obtain the swap file using \func{IomuGetSwapFile}, once you have the \textit{FILE\_OBJECT}
structure you can simply use the \func{IoReadFile} and \func{IoWriteFile} functions to read and
write from the swap FS. The only restriction is that the number of bytes transferred must be
exactly the size of a page and the offset must also be page aligned.

The size of the swap file is 128 MiB which should be sufficient for all the current tests (as long
as cleanup properly occurs after process termination): at most 80 MiB will be used at a single time.

\section{Assignment}

\subsection{Per Process Quotas}

You must implement a mechanism to keep track of the number of files currently held open by a process 
and the number of physical frames it currently uses. These definitions are found in \file{process\_internal.h}
and are \macro{PROCESS\_MAX\_PHYSICAL\_FRAMES} and \macro{PROCESS\_MAX\_OPEN\_FILES} both currently
defined as 16.

In case of a real OS the frames occupied by the binary would also add up to the physical frame quota,
however, due to the way in which \projectname loads the applications in memory it would be very hard
to implement this. As a result you will only have to count the frames allocated as a result of calls
to \func{SyscallVirtualAlloc} and the frames occupied by the user stack.

When the quota for files open is reached then that process should not be able to open any additional
files until it closes another file.

When the quota for physical frames is reached the eviction mechanism must be invoked, which will
pick one of the processes frames, swap it to disk and use it for another virtual memory allocation.

\subsection{System calls}

You need to implement the \func{SyscallVirtualAlloc} and \func{SyscallVirtualFree} system calls,
these allow user applications to allocate virtual memory in their address space.

Most of the parameters are identical to their kernel counterparts: \func{VmmAllocRegionEx} and
\func{VmmFreeRegionEx}. However, providing a MDL is not supported and instead of directly providing
a FILE\_OBJECT structure these system calls take a UM\_HANDLE to represent the file to map.

Also, you need to ensure that the user application is not allowed to specify both writable and
executable rights to a memory region. When this happens the system call should fail.

Also, the system call has an extra parameter, this is what allows the creation of shared memory (allows
user applications to share data). This extra parameter is called \var{Key} and has the following
semantics:
\begin{itemize}
	\item If the value is 0, the virtual allocation created is private for the current process.

	\item If the value is non-zero then the memory backed by the virtual allocation received by the
creator process can be accessed from any other process in the system. This is done by the second
process calling \func{SyscallVirtualAlloc} specifying the same \var{Key} value as the creator. This
means that \var{Key} acts as a global identifier which can be used by any other process in the
system. An illustration is shown in \fullref{lst:SharedMemory}.

	Assumming Process 0 starts execution first, when it calls \func{SyscallVirtualAlloc} the OS will
see that \var{Key} has a value different from 0 and will check to see if there is already an
allocation made with that key. Once it sees that there is none it will allocate a virtual range of
a page size with write access. Because the process also specified a \var{Key} value different from 0
the OS will also keep track of this allocation.

	When Process 1 gets to the \func{SyscallVirtualAlloc} call the OS will check again if there is
already an allocation using this key, it will find that there is already one previously created.
Because of this the OS will use the same backing physical frames for the virtual range returned to
the second process.

	Because both virtual allocations (in each process) share the same frames the second process will
display the string written by the first process.

\end{itemize}

\begin{lstlisting}[caption={Shared Memory},label={lst:SharedMemory}]
#define SHARED_KEY_VALUE	0x371

// Code executed by Process 0
void Process0Code(void)
{
	STATUS status;
	char* pData;
	
	status = SyscallVirtualAlloc(NULL, 
	PAGE_SIZE, 
	VMM_ALLOC_TYPE_RESERVE | VMM_ALLOC_TYPE_COMMIT,
	PAGE_RIGHTS_READWRITE,
	NULL,
	SHARED_KEY_VALUE,
	&pData);
	ASSERT(SUCCEEDED(status));
	
	strcpy(pData, "Shared memory is the coolest thing ever!");
	
	// ... Other code ...
}

// Code executed by Process 1
void Process1Code(void)
{
	STATUS status;
	char* pData;
	
	status = SyscallVirtualAlloc(NULL, 
	PAGE_SIZE, 
	VMM_ALLOC_TYPE_RESERVE | VMM_ALLOC_TYPE_COMMIT,
	PAGE_RIGHTS_READWRITE,
	NULL,
	SHARED_KEY_VALUE,
	&pData);
	ASSERT(SUCCEEDED(status));
	
	LOG("Data found in shared buffer is [%s]\n", pData);
	
	// ... Other code ...
}
\end{lstlisting}

\subsection{Swapping}

As previously mentioned, once a process reaches its quota for number of physical frames allocated
the contents of a frame must be swapped out to disk.

After the contents have been swapped out, another virtual address may be mapped into this physical
frame. From the point of view of a user-application this is transparent, the application only
works with virtual addresses and has no idea where the physical memory which actually holds the
data resides.

After you implemented the swap out operation be sure to also add support for swap in, else when the
application accesses the virtual address corresponding to the frame previously swapped out the
kernel won't be able to solve the \#PF exception. For this you will probably need to make changes
in \func{VmmSolvePageFault} and use the supplemental page table to determine the location of the
data on the swap file.

\subsection{Zero Pages}

You will also need to implement support for allocating pages initialized to zero. For this you will
need to modify \func{VmmAllocRegionEx}, remove the assert at line 543 and provide your own
implementation:
\begin{verbatim}
ASSERT(!IsBooleanFlagOn(AllocType, VMM_ALLOC_TYPE_ZERO));
\end{verbatim}

The trivial solution would be to simply mark that the page contains zeroes and when a page fault
occurs to memzero the memory, however there are more elegant and efficient ways of implementing
zero pages.

One of the tests actually checks the following scenario: allocate 2 GiB of virtual memory and read
the data from each page, you should remember that the swap file is only 128 MiB and a process is
restricted to 16 frames of physical memory at a time.

\subsection{Stack Growth}

Implement stack growth. In project 2, the stack was 8 pages allocated eagerly and programs were limited to that much stack.

Now, you should start with a stack of only 1 page and if the stack grows past its current size, allocate additional pages as necessary.

Allocate additional pages only if they "appear" to be stack accesses. Devise a heuristic
that attempts to distinguish stack accesses from other accesses. The x86 PUSH instruction checks access permissions before
it adjusts the stack pointer, so it may cause a page fault 8 bytes below the stack pointer.
(Otherwise, PUSH would not be restartable in a straightforward fashion.)

Within a page fault generated, you can retrieve the current stack pointer
from the \var{Rsp} member of \struct{INTERRUPT\_STACK\_COMPLETE} passed to \func{\_IsrExceptionHandler}
You should impose some absolute limit on stack size, as do most OSes.  Some OSes make
the limit user-adjustable, e.g. with the ulimit command on many Unix systems.  On many
GNU/Linux systems, the default limit is 8 MB. The first stack page need not be allocated lazily. You can allocate and initialize it with
the command line arguments at load time, with no need to wait for it to be faulted in.

Stack pages do not count towards the processes's frame quota.

\section{Source Files}

For this project most of your work will be done in the \textit{core/memory} filter and some will
be done in \textit{usermode}. Here's a quick overview of the files you'll be interested in:

\file{vmm.c}

Implements the page fault handler in \func{VmmSolvePageFault}.

\file{vm\_reservation\_space.c}

Works with the \textit{VMM\_RESERVATION\_SPACE} structures (a.k.a. supplemental page table entries).

\file{syscall.c}

Provides the system call dispatcher \func{SyscallHandler}, you will need to implement the two system
calls here.

\file{iomu.h}

Provides a pointer to an file instance to the swap file: \func{IomuGetSwapFile}.

\file{io.h}

Provides the IO functions for working with files: \func{IoFileRead} and \func{IoFileWrite}, you will
need these for working with the swap file. You do not need to open or close the file, this is done
by \projectname.

\section{FAQ}

\textbf{Q: Do we need a working Project 2 to implement Project 3?}

A: Yes.

\newline

\textbf{Q: How do we resume a process after we have handled a \#PF?}

A: Returnng from \func{VmmSolvePageFault} with TRUE resumes the current user process (see
\fullref{sect:PfHandling}). It will then retry the instruction to which the instruction pointer
points.

\newline

\textbf{Q: When sharing memory does \func{SyscallVirtualAlloc} need to return the same virtual
address for both processes?}

A: No, the value of the virtual address returned is irrelevant as long as it is backed up by the
same physical frames.

\newline

\textbf{Q: Can we increase the size of the swap file?}

A: Yes, you can, but you \textbf{MUST NOT} do so. The size was chosen as 128 MiB in such a way to
accommodate all the tests requirements and to require an efficient implementation for zero pages.